\setcounter{reaction}{0}%


\chapter{Background and Related Works} \label{chapter2}

This chapter presents the underlying concepts discussed in this thesis. In the first step, the concept of a digital platform is defined which a theoretical framework for the current project. In the second step, the attention focuses on the research literature about so call Governments as a Platform (GaaP) concept which has direct relevance with the project. Lastly, a quick overview of TOGAF framework is given as it is used for the presentation of the final results.

\section{Digital Platforms}
The concept of a digital platform is a complicated research subject not only because of its omnipresence in today's industries, but also because its multidisciplinary nature \citep{deReuver:2018}. Initially, the concept of a platform was not necessarily binded with digital innovation and digital platforms. The focus was mainly on the economic and business side of the process. A platform was defined as a stable core and a variable periphery <Baldwin and Woodard, 2009> which was primarily realized through modularisation <Henderson and Clark, 1990; Baldwin and Clark, 2000>. The concepts of two-sided markets, network effects, multi-sided platforms, etc. became the top importance.

\section{Government as a Platform}

\section{TOGAF}

